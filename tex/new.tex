\section{A New Platform, Hardware and Software}

The first hardware design for the new platform, called \namedevice, has a
Cortex~M4 processor with many advanced features (memory protection,
digital signal processing instructions).  To support connectivity with
both the established low-power wireless technology and emerging PANs,
\namedevice has both 802.15.4 and BLE radios.  Recognizing both the
importance that fine-grained power measurement and control has played
in previous work and the pain points for performing power research in
previous designs, \namedevice will incorporate a highly instrumented and
flexible power network.

The increased complexity of peripherals, power management, and clock
domains and the needs of new applications
motivate a new operating system, called \nameos. The \nameos kernel builds
on the past decade of research to provide stable, simple implementations of
core abstractions such as IPv6, sensing, and local storage. 
The core idea and abstraction in \nameos is time -- global time for timestamps,
local time that stretches and shrinks in response to energy availability, and
clock source management for efficiency and performance.
The operating system will enable mechanisms for emerging embedded applications,
including opportunistic and privacy-preserving data muling and sharing.

In summary, we aim to create a foundation for a new generation of
groundbreaking research by capitalizing on a decade of technological
improvements and on knowledge gleaned from fifteen years of
networked embedded systems research.




%%%%%%%%%%%%%%%%%%%%%%%%%%%%%%%%%%%%%%%%%%%%%%%%%%%%%%%%%%%%%%%%%%%%%%%%%%%%%%%%
\begin{comment}

%% The next generation mote will reflect the diversity and flexibility that comes
%% with a more mature field. Its processor will be more capable with more memory,
%% yet at an equal or lower power budget. Its radio will be more flexible,
%% incorporating basic SDR-like features and exposing more control over link and
%% MAC operation. While wired I/O has not changed much yet in recent years,
%% next-next generation hardware anticipates a need for a new
%% interconnect~\cite{kuo14mbus}, and looking forward towards these kind of
%% interfaces will aide long-term viability.

Time and experience have identified fundamental concepts to re-think and
re-build for the next generation embedded operating
system~\cite{tinyos-retrospective}. Some key needed improvements are
i) reducing the developer learning curve, potentially
incorporating non-traditional design tools to assist
programmers, %~\cite{brown-bubbles},
ii) adding advanced constructs that enable sequential reasoning in an 
event-driven model, iii) developing
better ``operating system''-like behavior, better isolating application
developers from resource contention and exposing interfaces that permit
higher-level application logic to be written in friendly, easier languages
(e.g. a syscall interface), iv) exploring new ways of programmatically
conceptualizing time to enable network and system performance to better match
programmer desire to system capability, v) expanding the operating system 
(or libraries)
to provide simple, feature complete utilities
like networking, without sacrificing the research community's capability to
explore, and vi) deepening community and industry involvement, starting with
building stronger ties and buy-in from people already invested in the embedded
ecosystem, looking towards enabling usability by true novices, much as
Arduino and other upcoming platforms do today.

% I do not like this conclucion
%
% Aaand... we don't have space for it anyway; that's one solution...
% The goal of this poster is to capture and spark a discussion towards solving
% many or all of these issues and to bring networked embedded systems into the
% post-mote era.




% (from gdoc):
%
% [as written, this misses some of the higher-level ideas from Phil’s e-mail;
% network bridging, advanced storage, peripheral management, etc]
% 
% MPA: I think point iii can be expanded. We want to isolate the user from the
% complexities of writing ‘correct’ event-driven software, and having decent
% communication capabilities - which I think was one of the original goals of
% TinyOS - but we also now want to isolate users from the complexities of
% dealing with advancing hardware. Users enjoy Arduino because you don’t need to
% fiddle with registers to get things to work, if we can offer a level of
% abstraction such that users can make low-energy applications on
% high-performance systems, it would be great.
%
% --> I think I address this better now across iii and iv
% 
% *Pat: This is something I’ve been thinking about for a long time. It makes
% writing event-driven code much easier to reason about and I think it is
% possible to build into something like this into TinyOS / nesC and I think I
% know how to do it. I would love to sit down and chat with someone about this
% some time.
% 
% 
% people don’t want to build custom hw all the time; mote needs to be future
% proof. Work today even uses telosBs because they can and they’re available
% (well, ish, harder to buy now).
% 

\end{comment}
