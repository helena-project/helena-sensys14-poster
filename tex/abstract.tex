% ABSTRACT

For the last fifteen years, research explored the hardware, software, sensing,
communication abstractions, languages, and protocols that could make networks
of small, embedded devices---motes---sample and report data for long periods
of time unattended.
%
Today, the
application and technological landscapes have shifted, introducing new
requirements and new capabilities. Hardware has evolved past 8~and 16~bit
microcontrollers: there are now 32~bit processors with lower energy budgets
and greater computing capability. New wireless link layers have
emerged, creating protocols that support rapid and efficient setup and
teardown but introduce novel limitations that systems must consider.
%
%Software and tools for developing efficient, reliable software for
%resource-constrained devices have also improved greatly.
%
The time has come to look beyond optimizing networks of motes. We look towards
new technologies such as Bluetooth Low Energy, Cortex~M processors, and
capable energy harvesting, with new application spaces such as personal area
networks, and new capabilities and requirements in security and privacy to
inform 
%the
contemporary
%
hardware and software platforms.  It is time for a new, open
experimental platform in this post-mote era.%\color{red}{*}

%\marginparwidth=35pt
%\marginpar{\color{red}{If we're proposing a platform, we should do it here!}}
