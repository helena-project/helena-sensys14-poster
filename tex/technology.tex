\section{Technology Today}

\smallskip\noindent
\textbf{Integrating TinyOS and Bluetooth Low Energy.}
The majority of BLE stacks provided by chip manufacturers expect them to be
implemented in embedded systems that follow some variation of the sequential
programming model. TinyOS is an event driven system designed for low power
embedded electronics. Recent work sought to adapt the Nordic BLE stack to
work in this environment and found that the quickest integration of these two required
making TinyOS's event driven programming model function sequentially. As TinyOS
was not designed to incorporate large sequential software abstractions this
resulted in numerous inefficiencies. While an event-driven stack written from
the ground up may be able to overcome these, finding a means to integrate
TinyOS with manufacturer-provided software stacks is a requirement to enable
more widespread adoption of the platform.

% The question remains open whether a sequential
% system with the same versatility as TinyOS could eliminate these
% inefficiencies.  Also, it is possible that a stack
% constructed from the ground up that leveraged event driven programming and BLE
% technology could result in an embedded system more efficient than the
% manufacturer provided software implementations.  BLE is an excellent
% technology to be used in embedded systems but integrating TinyOS with stacks
% provided by manufactures is not sustainable nor ideal for future research.

\smallskip\noindent
\textbf{Cortex M0--M4.}
Low energy ARM Cortex M0--M4 MCUs provide a new alternative for
networked embedded systems. They are significantly more computationally
powerful and while the appearance of a more powerful processor is not
particularly surprising, it may significantly expand the domain of problems
that can be solved by a `mote'.
%
These MCUs
feature notably more peripherals than previous generation 8~or 16~bit
MCUs and can offer elegant solutions to application problems
or dramatically increase performance per joule. To still meet low
power targets, however, these MCUs have very complex power
management and peripheral configuration mechanisms. While these are fairly
easy to configure and abstract if energy usage is not a concern, as in most
Arduino-type systems, optimal energy usage requires a very close coupling
between the dynamically changing application states and the peripheral
configuration. This close coupling is difficult to reconcile with the goal of
an embedded operating system that presents a simple environment for powerful
application programming to novice users.

% Interesting and fair point, but more on the systems side than the MCU, which
% is the discussion here (and we only have 2 pages..)
%
% A final characteristic is that the increase in DMA channels combined with the
% ability to configure peripheral events to asynchronously trigger other
% peripherals and DMA transfers allows experienced programmers to implement
% important parts of application logic with no CPU cycles - which can
% drastically reduce the time that the CPU is spent awake. The configuration of
% such functionality is fairly hardware specific and the question of how to
% automatically use this functionality or present it to novice programmers in an
% easy to use form is challenging.

\smallskip\noindent
\textbf{Impact of Energy Harvesting.}
Today, it is possible to obtain temperature, humidity, or light levels every
minute or two, under dimly lit conditions, using indoor photovoltaics that
occupy a mere 2--3~cm$^2$.
%It is possible to build an energy-harvesting energy meter that measures
%real power in a form factor no larger than a clip-on split-core current
%transformer.
A decade of research in energy-harvesting transducers, efficient power
conversion electronics, and compact energy storage technologies, coupled
with reductions in the idle and active power of radios,
instant-on/instant-write phase-change memory, and modern MCUs make compact
energy-harvesting sensors that can run perpetually, even in many indoor
settings, a reality.
%
However, writing applications for such intermittently-powered,
energy-harvesting systems is a challenge.  It requires revisiting many
assumptions about system startup, maintaining state across activation cycles,
discovering and communicating with neighbors, predicting future energy
availability, and scheduling operations under energy uncertainty.  One way to
circumvent some of the challenges is to grossly over provision the energy
harvesting power supply in an attempt to mask the energy intermittency and all
that dealing with it entails.  But this drives up size and cost, and it cannot
fully mitigate the variability inherent in energy-harvesting operation.  A
different path exposes the variability, perhaps through richer power supply
interfaces, to the system software which can then context switch on time or
energy quanta, and manage continuations across activations by leveraging fast
and efficient context storage and restoration.

