\section{Applications Today}

A decade ago, stationary fixed rate (or occasionally event-driven) data 
reporting applications dominated sensor networks and their research. 
New applications that incorporate mobility, hybrid networking, and 
personal privacy are emerging. A new platform should embrace and enable 
these new opportunities.

\smallskip\noindent
\textbf{Human-centric.}
The mobile device is a gateway to an embedded network. Instead of
sending data to a fixed collection point, the network can send directly
to a mobile device, on demand. This enables interaction with
surrounding devices and infrastructure without needing to know
URLs or logging into a cloud system. Proximity networks provide a degree
of basic, physical security that enables opportunistic interactions.
%You can walk into a room, turn on its lights, ask it how long it is free for,
%and see what other rooms are free.

Because these interactions are human-centric, they require low latency and
are highly bursty -- two performance properties unimportant in the
dominant network model of the past decade. These low-latency interactions
are more than event notifications or alarms. They can involve significant
queries and exchanges of data. Furthermore, because they involve mobile
devices, existing approaches of long-term link estimation are of limited use.
These new application requirements require new communication abstractions,
rethinking the tradeoffs between latency, storage, energy, snooping, and how
an OS will support them.

\smallskip\noindent
\textbf{Perpetual Networks.}
Energy harvesting and low-power peripherals will finally enable a
long-term goal of sensor networks: perpetual networks. Imagine
iBeacons and smoke detectors that need no battery replacements. If the world
will be filled with thousands of smart objects per person, energy must recede
to be a non-issue for almost all of them.

We cannot predict the performance of a solar cell. This means an OS
is stuck between two big unknowns: the future energy available as well as
the potential energy needs from bursty, human-centric interactions. For
an embedded device to be truly perpetual, there must be platform support to
scale behavior and performance based on these two factors, optimizing needs,
wants, and energy use.

\smallskip\noindent
\textbf{Privacy and Proximity in Networks.}
Interactions between PANs (centered around a user’s mobile phone)
and proximity applications such as iBeacons happen in public, with
never-seen-before peers. This presents a dual security and privacy problem. On
the one hand, connections between the PAN and proximity device must be
confidential and authentic -- e.g. payments.
On the other hand, casual interactions with
proximity devices must not enable ubiquitous tracking of users. Unfortunately,
confidential and authentic communication and anonymity are difficult to
achieve simultaneously.
%
Operating systems can play a role in addressing these issues, for example by coordinating
security features in the BLE stack with application specific knowledge.

%% Too detailed an example for a poster abstract
%
% For example, we propose a new mechanism -- Address Isolation -- that leverages
% Resolvable Private Addresses in BLE to expose multiple private identities from
% the same end-user device. Resolable Private Addresses are BLE device addresses
% derived from a public random value and a shared secret between peers. This
% allows peers to identify each other through addresses that appear random to
% nearby devices that do not share the secret. Address Isolation uses this BLE
% mechanism to expose multiple identities (each resolvable by, e.g. a single
% point-of-sale operator, but anonymous to everyone else) at the OS level, by
% coordinating when each identity should be used with applications. For example,
% in the point-of-sale scenario, the application loads a shared secret key with
% a point-of-sale operator (e.g. Wal-Mart) at application install time, and
% tells the operating system to use the identity based on that key upon user
% interaction, such as initiating a payment.

