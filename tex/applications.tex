\section{Applications Today}

\smallskip\noindent
\textbf{How Applications Drive Platforms.}

\glipsum[1-2]

\smallskip\noindent
\textbf{Privacy and Proximity in Networks.}
Interactions between personal area networks (centered around a user’s mobile
phone) and proximity applications such as iBeacons happen in public, with
never-seen-before peers. This presents a dual security and privacy problem. On
the one hand, connections between the PAN and proximity device must be
confidential and authentic -- e.g., in a point-of-sale application, the person
behind me in line should not be able to read my credit-card information or
make purchases on my behalf. On the other hand, casual interactions with
proximity devices must not enable ubiquitous tracking of users. Unfortunately,
confidential and authentic communication and anonymity are difficult to
achieve simultaneously.

Operating systems can play a role in addressing these issues by coordinating
security features in the BLE stack with application specific knowledge. For
example, we propose a new mechanism -- Address Isolation -- that leverages
Resolvable Private Addresses in BLE to expose multiple private identities from
the same end-user device. Resolable Private Addresses are BLE device addresses
derived from a public random value and a shared secret between peers. This
allows peers to identify each other through addresses that appear random to
nearby devices that do not share the secret. Address Isolation uses this BLE
mechanism to expose multiple identities (each resolvable by, e.g. a single
point-of-sale operator, but anonymous to everyone else) at the OS level, by
coordinating when each identity should be used with applications. For example,
in the point-of-sale scenario, the application loads a shared secret key with
a point-of-sale operator (e.g. Wal-Mart) at application install time, and
tells the operating system to use the identity based on that key upon user
interaction, such as initiating a payment.

