\section{Introduction}
\label{sec:intro}

Over a decade ago, a flurry of hardware platforms and supporting software
empowered the research community to investigate and explore wireless sensor
networks and their applications. Many projects today still use these
``mote''-class devices, exploring problems in systems software, low-power
networking, and application design.
%
The underlying technology and its potential applications have progressed a
great deal in the past decade. The Cortex~M series of 32-bit processors
finally have sleep currents competitive with MSP430 and AVR microcontrollers
(MCUs).\footnote{
  e.g.\ the NXP LPC11U6X family draws 1~\uA in sleep with active RTC.
}
802.15.4 has grown far beyond the closed world of ZigBee, with
new physical layers for new applications. The recent incorporation of
Bluetooth Low Energy (BLE) into mobile phones allows ubiquitous sensing networks to
directly interact with human-centric devices. Also driven by phones, sensors
themselves are orders of magnitude more energy efficient and precise.

Simultaneously, applications have become much richer. Applications in early
sensor network research focused on fixed rate, long-term sensing, guiding a
research agenda of ultra-low power operation and robust multi-hop networking.
Today, personal area networks (PANs), tether to phones and interact with
proximity networks such as iBeacon; building-area networks share knowledge, like
occupancy, among security, HVAC, and lighting control.
In addition, the rise of ``maker culture'' and their platforms and communities
(e.g.\ Arduino~\cite{arduino}) has led to a level of diversity and
accessibility that early research platforms simply could not provide.

We have an explosion of new applications and developers, each with new and challenging
requirements. We have reached a turning point in hardware, enabling a whole
new class of device and operational models. It is time for a new OS and family
of hardware platforms to explore and research embedded networked systems in
the post-mote era.

